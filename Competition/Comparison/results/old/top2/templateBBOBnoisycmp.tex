% This is based on "sig-alternate.tex" V1.9 April 2009
% This file should be compiled with V2.4 of "sig-alternate.cls" April 2009
%
% ----------------------------------------------------------------------------------------------------------------
%
% This .tex source is an example which *does* use
% the .bib file (from which the .bbl file % is produced).
% REMEMBER HOWEVER: After having produced the .bbl file,
% and prior to final submission, you *NEED* to 'insert'
% your .bbl file into your source .tex file so as to provide
% ONE 'self-contained' source file.
%
% Information on the sig-alternate class file and on the
% GECCO workshop paper format and submission can be found at these
% locations:
% http://www.acm.org/sigs/publications/proceedings-templates#aL2
% http://www.sheridanprinting.com/typedept/gecco3.htm
%
% ================= IF YOU HAVE QUESTIONS =======================
% Questions regarding the SIGS styles, SIGS policies and
% procedures, Conferences etc. should be sent to
% Adrienne Griscti (griscti@acm.org)
%
% Technical questions to bbob@lri.fr
% ===============================================================
%

\documentclass{sig-alternate}
\pdfpagewidth=8.5in
\pdfpageheight=11in
\special{papersize=8.5in,11in}
    \renewcommand{\topfraction}{1}  % max fraction of floats at top
    \renewcommand{\bottomfraction}{1} % max fraction of floats at bottom
    %   Parameters for TEXT pages (not float pages):
    \setcounter{topnumber}{3}
    \setcounter{bottomnumber}{3}
    \setcounter{totalnumber}{3}     % 2 may work better
    \setcounter{dbltopnumber}{4}    % for 2-column pages
    \renewcommand{\dbltopfraction}{1}   % fit big float above 2-col. text
    \renewcommand{\textfraction}{0.0}   % allow minimal text w. figs
    %   Parameters for FLOAT pages (not text pages):
    \renewcommand{\floatpagefraction}{0.80} % require fuller float pages
    % N.B.: floatpagefraction MUST be less than topfraction !!
    \renewcommand{\dblfloatpagefraction}{0.7}   % require fuller float pages

%%%%%%%%%%%%%%%%%%%%%%%%%%%%%%%%%%%%%%%%%%%%%%%%%%%%%%
% Packages
\usepackage{graphicx}
\usepackage{xcolor}
\usepackage{float}
\usepackage{rotating}


%%%%%%%%%%%%%%%%%%%%%%%%%%%%%%%%%%%%%%%%%%%%%%%%%%%%%%
% Definitions

% location of pictures files
\newcommand{\bbobdatapath}{cmp2data/}
\graphicspath{{\bbobdatapath}}

% pre-defined commands
\newcommand{\DIM}{\ensuremath{\mathrm{DIM}}}
\newcommand{\ERT}{\ensuremath{\mathrm{ERT}}}
\newcommand{\FEvals}{\ensuremath{\mathrm{FEvals}}}
\newcommand{\nruns}{\ensuremath{\mathrm{Nruns}}}
\newcommand{\Dfb}{\ensuremath{\Delta f_{\mathrm{best}}}}
\newcommand{\Df}{\ensuremath{\Delta f}}
\newcommand{\nbFEs}{\ensuremath{\mathrm{\#FEs}}}
\newcommand{\fopt}{\ensuremath{f_\mathrm{opt}}}
\newcommand{\ftarget}{\ensuremath{f_\mathrm{t}}}
\newcommand{\CrE}{\ensuremath{\mathrm{CrE}}}

% specify acronyms for algorithm_0 (1st arg. of post-processing) and algorithm_1 (2nd arg.) 
% acronyms should be shorter than 10 characters
\newcommand{\algzero}{ALG0-acronym}  % first argument in the post-processing
\newcommand{\algone}{ALG1-acronym}   % second argument in the post-processing

%%%%%%%%%%%%%%%%%%%%%%%%%%%%%%%%%%%%%%%%%%%%%%%%%%%%%%

\begin{document}

%
% --- Author Metadata here ---
\conferenceinfo{GECCO'10,} {July 7--11, 2010, Portland, Oregon, USA.}
\CopyrightYear{2010}
\crdata{978-1-4503-0073-5/10/07}
\clubpenalty=10000
\widowpenalty = 10000
% --- End of Author Metadata ---

\title{Black-Box Optimization Benchmarking Template for the Comparison of Two Algorithms on the Noisy Testbed}
\subtitle{Draft version
\titlenote{Submission deadline: March 25th.}}
%Camera-ready paper due April 13th.}}

%
% You need the command \numberofauthors to handle the 'placement
% and alignment' of the authors beneath the title.
%
% For aesthetic reasons, we recommend 'three authors at a time'
% i.e. three 'name/affiliation blocks' be placed beneath the title.
%
% NOTE: You are NOT restricted in how many 'rows' of
% "name/affiliations" may appear. We just ask that you restrict
% the number of 'columns' to three.
%
% Because of the available 'opening page real-estate'
% we ask you to refrain from putting more than six authors
% (two rows with three columns) beneath the article title.
% More than six makes the first-page appear very cluttered indeed.
%
% Use the \alignauthor commands to handle the names
% and affiliations for an 'aesthetic maximum' of six authors.
% Add names, affiliations, addresses for
% the seventh etc. author(s) as the argument for the
% \additionalauthors command.
% These 'additional authors' will be output/set for you
% without further effort on your part as the last section in
% the body of your article BEFORE References or any Appendices.

\numberofauthors{1} %  in this sample file, there are a *total*
% of EIGHT authors. SIX appear on the 'first-page' (for formatting
% reasons) and the remaining two appear in the \additionalauthors section.
%
\author{
% You can go ahead and credit any number of authors here,
% e.g. one 'row of three' or two rows (consisting of one row of three
% and a second row of one, two or three).
%
% The command \alignauthor (no curly braces needed) should
% precede each author name, affiliation/snail-mail address and
% e-mail address. Additionally, tag each line of
% affiliation/address with \affaddr, and tag the
% e-mail address with \email.
%
% 1st. author
\alignauthor
Forename Name\\ %\titlenote{Dr.~Trovato insisted his name be first.}\\
%       \affaddr{Institute for Clarity in Documentation}\\
%       \affaddr{1932 Wallamaloo Lane}\\
%       \affaddr{Wallamaloo, New Zealand}\\
%       \email{trovato@corporation.com}
%% 2nd. author
%\alignauthor
%G.K.M. Tobin\titlenote{The secretary disavows
%any knowledge of this author's actions.}\\
%       \affaddr{Institute for Clarity in Documentation}\\
%       \affaddr{P.O. Box 1212}\\
%       \affaddr{Dublin, Ohio 43017-6221}\\
%       \email{webmaster@marysville-ohio.com}
%% 3rd. author
%\alignauthor Lars Th{\o}rv{\"a}ld\titlenote{This author is the
%one who did all the really hard work.}\\
%       \affaddr{The Th{\o}rv{\"a}ld Group}\\
%       \affaddr{1 Th{\o}rv{\"a}ld Circle}\\
%       \affaddr{Hekla, Iceland}\\
%       \email{larst@affiliation.org}
%\and  % use '\and' if you need 'another row' of author names
%% 4th. author
%\alignauthor Lawrence P. Leipuner\\
%       \affaddr{Brookhaven Laboratories}\\
%       \affaddr{Brookhaven National Lab}\\
%       \affaddr{P.O. Box 5000}\\
%       \email{lleipuner@researchlabs.org}
%% 5th. author
%\alignauthor Sean Fogarty\\
%       \affaddr{NASA Ames Research Center}\\
%       \affaddr{Moffett Field}\\
%       \affaddr{California 94035}\\
%       \email{fogartys@amesres.org}
%% 6th. author
%\alignauthor Charles Palmer\\
%       \affaddr{Palmer Research Laboratories}\\
%       \affaddr{8600 Datapoint Drive}\\
%       \affaddr{San Antonio, Texas 78229}\\
%       \email{cpalmer@prl.com}
} % author
%% There's nothing stopping you putting the seventh, eighth, etc.
%% author on the opening page (as the 'third row') but we ask,
%% for aesthetic reasons that you place these 'additional authors'
%% in the \additional authors block, viz.
%\additionalauthors{Additional authors: John Smith (The Th{\o}rv{\"a}ld Group,
%email: {\texttt{jsmith@affiliation.org}}) and Julius P.~Kumquat
%(The Kumquat Consortium, email: {\texttt{jpkumquat@consortium.net}}).}
%\date{30 July 1999}
%% Just remember to make sure that the TOTAL number of authors
%% is the number that will appear on the first page PLUS the
%% number that will appear in the \additionalauthors section.

\maketitle
\begin{abstract}
\end{abstract}

% Add any ACM category that you feel is needed
\category{G.1.6}{Numerical Analysis}{Optimization}[global optimization,
unconstrained optimization]
\category{F.2.1}{Analysis of Algorithms and Problem Complexity}{Numerical Algorithms and Problems}

% Complete with anything that is needed
\terms{Algorithms}

% Complete with anything that is needed
\keywords{Benchmarking, Black-box optimization}

% \section{Introduction}
%
% \section{Algorithm Presentation}
%
% \section{Experimental Procedure}
%
%%%%%%%%%%%%%%%%%%%%%%%%%%%%%%%%%%%%%%%%%%%%%%%%%%%%%%%%%%%%%%%%%%%%%%%%%%%%%%%
\section{Results}
%%%%%%%%%%%%%%%%%%%%%%%%%%%%%%%%%%%%%%%%%%%%%%%%%%%%%%%%%%%%%%%%%%%%%%%%%%%%%%%

Results from experiments according to \cite{hansen2010exp} on the benchmark
functions given in \cite{wp200902_2010,hansen2010noi} are presented in
Figures~\ref{fig:ERTratiographs}, \ref{fig:scatterplots} and \ref{fig:RLDs} and
in Table~\ref{tab:ERTs}. The \textbf{expected running time (ERT)}, used in the figures and table, depends on a
given target function value, $\ftarget=\fopt+\Delta\ftarget$, and is computed over all relevant trials
as the number of function evaluations executed during each trial while the best
function value did not reach \ftarget, summed over all trials
and divided by the number of trials that actually reached \ftarget\
\cite{hansen2010exp,price1997dev}. 
\textbf{Statistical significance} is tested with the rank-sum test for a given
target $\Delta\ftarget$ ($10^{-8}$ in Figure~\ref{fig:ERTratiographs}) using,
for each trial, either the number of needed function evaluations to reach
$\Delta\ftarget$ (inverted and multiplied by $-1$), or, if the target was not
reached, the best $\Df$-value achieved, measured only up to the smallest number
of overall function evaluations for any unsuccessful trial under consideration. 

%%%%%%%%%%%%%%%%%%%%%%%%%%%%%%%%%%%%%%%%%%%%%%%%%%%%%%%%%%%%%%%%%%%%%%%%%%%%%%%
\begin{figure*}
\centering
\begin{tabular}{@{}c@{}c@{}c@{}c@{}c@{}}
\includegraphics[width=0.204\textwidth, trim= 0.7cm 0.8cm 0.5cm 0.5cm, clip]{ppcmpfig_f101}&
\includegraphics[width=0.193\textwidth, trim= 1.8cm 0.8cm 0.5cm 0.5cm, clip]{ppcmpfig_f104}&
\includegraphics[width=0.193\textwidth, trim= 1.8cm 0.8cm 0.5cm 0.5cm, clip]{ppcmpfig_f107}&
\includegraphics[width=0.193\textwidth, trim= 1.8cm 0.8cm 0.5cm 0.5cm, clip]{ppcmpfig_f110}&
\includegraphics[width=0.193\textwidth, trim= 1.8cm 0.8cm 0.5cm 0.5cm, clip]{ppcmpfig_f113}\\
\includegraphics[width=0.204\textwidth, trim= 0.7cm 0.8cm 0.5cm 0.5cm, clip]{ppcmpfig_f102}&
\includegraphics[width=0.193\textwidth, trim= 1.8cm 0.8cm 0.5cm 0.5cm, clip]{ppcmpfig_f105}&
\includegraphics[width=0.204\textwidth, trim= 1.8cm 0.8cm 0.5cm 0.5cm, clip]{ppcmpfig_f108}&
\includegraphics[width=0.193\textwidth, trim= 1.8cm 0.8cm 0.5cm 0.5cm, clip]{ppcmpfig_f111}&
\includegraphics[width=0.193\textwidth, trim= 1.8cm 0.8cm 0.5cm 0.5cm, clip]{ppcmpfig_f114}\\
\includegraphics[width=0.204\textwidth, trim= 0.7cm 0.8cm 0.5cm 0.5cm, clip]{ppcmpfig_f103}&
\includegraphics[width=0.193\textwidth, trim= 1.8cm 0.8cm 0.5cm 0.5cm, clip]{ppcmpfig_f106}&
\includegraphics[width=0.193\textwidth, trim= 1.8cm 0.8cm 0.5cm 0.5cm, clip]{ppcmpfig_f109}&
\includegraphics[width=0.193\textwidth, trim= 1.8cm 0.8cm 0.5cm 0.5cm, clip]{ppcmpfig_f112}&
\includegraphics[width=0.193\textwidth, trim= 1.8cm 0.8cm 0.5cm 0.5cm, clip]{ppcmpfig_f115}\\\hline
\includegraphics[width=0.204\textwidth, trim= 0.7cm 0.8cm 0.5cm 0.5cm, clip]{ppcmpfig_f116}&
\includegraphics[width=0.193\textwidth, trim= 1.8cm 0.8cm 0.5cm 0.5cm, clip]{ppcmpfig_f119}&
\includegraphics[width=0.193\textwidth, trim= 1.8cm 0.8cm 0.5cm 0.5cm, clip]{ppcmpfig_f122}&
\includegraphics[width=0.193\textwidth, trim= 1.8cm 0.8cm 0.5cm 0.5cm, clip]{ppcmpfig_f125}&
\includegraphics[width=0.193\textwidth, trim= 1.8cm 0.8cm 0.5cm 0.5cm, clip]{ppcmpfig_f128}\\
\includegraphics[width=0.204\textwidth, trim= 0.7cm 0.8cm 0.5cm 0.5cm, clip]{ppcmpfig_f117}&
\includegraphics[width=0.193\textwidth, trim= 1.8cm 0.8cm 0.5cm 0.5cm, clip]{ppcmpfig_f120}&
\includegraphics[width=0.193\textwidth, trim= 1.8cm 0.8cm 0.5cm 0.5cm, clip]{ppcmpfig_f123}&
\includegraphics[width=0.193\textwidth, trim= 1.8cm 0.8cm 0.5cm 0.5cm, clip]{ppcmpfig_f126}&
\includegraphics[width=0.193\textwidth, trim= 1.8cm 0.8cm 0.5cm 0.5cm, clip]{ppcmpfig_f129}\\
\includegraphics[width=0.204\textwidth, trim= 0.7cm 0.0cm 0.5cm 0.5cm, clip]{ppcmpfig_f118}&
\includegraphics[width=0.193\textwidth, trim= 1.8cm 0.0cm 0.5cm 0.5cm, clip]{ppcmpfig_f121}&
\includegraphics[width=0.193\textwidth, trim= 1.8cm 0.0cm 0.5cm 0.5cm, clip]{ppcmpfig_f124}&
\includegraphics[width=0.193\textwidth, trim= 1.8cm 0.0cm 0.5cm 0.5cm, clip]{ppcmpfig_f127}&
\includegraphics[width=0.193\textwidth, trim= 1.8cm 0.0cm 0.5cm 0.5cm, clip]{ppcmpfig_f130}
\end{tabular}
% \vspace*{-0.2cm}
\caption{\label{fig:ERTratiographs}
Ratio of the expected running times (ERT) 
of \algone\ divided by \algzero\ versus $\log_{10}(\Df)$ for 
$f_{101}$--$f_{130}$ in
  {\color{cyan}2},
  {\color{green!45!black}3},
  {\color{blue}5}, 
  {\color{black}10}, 
  {\color{red}20}, 
  {\color{magenta}40}-D.
% cyan:2, green:3, blue:5, black:10, red:20, purple:40. 
Ratios $<10^0$ indicate an advantage of \algone, smaller
values are always better. The line gets dashed when for any algorithm the ERT exceeds thrice the median
of the trial-wise overall number of $f$-evaluations for the same algorithm on this function.
Symbols indicate the best achieved $\Df$-value of one algorithm (ERT gets undefined to the right). The dashed line
continues as the fraction of successful trials of the other
algorithm, where 0 means 0\% and the y-axis limits mean 100\%, values below
zero for \algone. The line ends when no algorithm reaches $\Df$ anymore. 
The number of successful trials is given, only if it was in $\{1\dots9\}$ for
\algone\ (1st number) and non-zero for \algzero\ (2nd number). 
Results are statistically significant with $p=0.05$ for one star and $p=10^{-\#\star}$
otherwise, with Bonferroni correction within each figure.}
% 
\end{figure*}
%%%%%%%%%%%%%%%%%%%%%%%%%%%%%%%%%%%%%%%%%%%%%%%%%%%%%%%%%%%%%%%%%%%%%%%%%%%%%%%
%%%%%%%%%%%%%%%%%%%%%%%%%%%%%%%%%%%%%%%%%%%%%%%%%%%%%%%%%%%%%%%%%%%%%%%%%%%%%%%
\begin{figure*}
\small
\centering
\begin{tabular}{@{}c@{}*{5}{@{}c@{}}}
 & {\sf 101 Sphere (moderate)} & {\sf 104 Rosenbrock (moderate)} & {\sf 107 Sphere} & {\sf 110 Rosenbrock} & {\sf 113 Step-ellipsoid} \\
\begin{turn}{90}\parbox{0.175\textwidth}{\centering\sf Gauss noise}\end{turn} &
    \includegraphics[height=0.175\textwidth, trim= 34mm 9mm 20mm 9mm, clip]{scatter_f101}&
    \includegraphics[height=0.175\textwidth, trim= 34mm 9mm 20mm 9mm, clip]{scatter_f104}&
    \includegraphics[height=0.175\textwidth, trim= 34mm 9mm 20mm 9mm, clip]{scatter_f107}&
    \includegraphics[height=0.175\textwidth, trim= 34mm 9mm 20mm 9mm, clip]{scatter_f110}&
    \includegraphics[height=0.175\textwidth, trim= 34mm 9mm 20mm 9mm, clip]{scatter_f113}\\
 & {\sf 102 Sphere (moderate)} & {\sf 105 Rosenbrock (moderate)} & {\sf 108 Sphere} & {\sf 111 Rosenbrock} & {\sf 114 Step-ellipsoid}\\
\begin{turn}{90}\parbox{0.175\textwidth}{\centering\sf uniform noise}\end{turn} &
    \includegraphics[height=0.175\textwidth, trim= 34mm 9mm 20mm 9mm, clip]{scatter_f102}&
    \includegraphics[height=0.175\textwidth, trim= 34mm 9mm 20mm 9mm, clip]{scatter_f105}&
    \includegraphics[height=0.175\textwidth, trim= 34mm 9mm 20mm 9mm, clip]{scatter_f108}&
    \includegraphics[height=0.175\textwidth, trim= 34mm 9mm 20mm 9mm, clip]{scatter_f111}&
    \includegraphics[height=0.175\textwidth, trim= 34mm 9mm 20mm 9mm, clip]{scatter_f114}\\
 & {\sf 103 Sphere (moderate)} & {\sf 106 Rosenbrock (moderate)} & {\sf 109 Sphere} & {\sf 112 Rosenbrock} & {\sf 115 Step-ellipsoid}\\
\begin{turn}{90}\parbox{0.175\textwidth}{\centering\sf Cauchy noise}\end{turn} &
    \includegraphics[height=0.175\textwidth, trim= 34mm 9mm 20mm 9mm, clip]{scatter_f103}&
    \includegraphics[height=0.175\textwidth, trim= 34mm 9mm 20mm 9mm, clip]{scatter_f106}&
    \includegraphics[height=0.175\textwidth, trim= 34mm 9mm 20mm 9mm, clip]{scatter_f109}&
    \includegraphics[height=0.175\textwidth, trim= 34mm 9mm 20mm 9mm, clip]{scatter_f112}&
    \includegraphics[height=0.175\textwidth, trim= 34mm 9mm 20mm 9mm, clip]{scatter_f115}\\\hline
 & {\sf 116 Ellipsoid} & {\sf 119 Sum of diff.\ powers} & {\sf 122 Schaffer F7} & {\sf 125 Griewank-Rosenbrock} & {\sf 128 Gallagher}\\
\begin{turn}{90}\parbox{0.175\textwidth}{\centering\sf Gauss noise}\end{turn} &
    \includegraphics[height=0.175\textwidth, trim= 34mm 9mm 20mm 9mm, clip]{scatter_f116}&
    \includegraphics[height=0.175\textwidth, trim= 34mm 9mm 20mm 9mm, clip]{scatter_f119}&
    \includegraphics[height=0.175\textwidth, trim= 34mm 9mm 20mm 9mm, clip]{scatter_f122}&
    \includegraphics[height=0.175\textwidth, trim= 34mm 9mm 20mm 9mm, clip]{scatter_f125}&
    \includegraphics[height=0.175\textwidth, trim= 34mm 9mm 20mm 9mm, clip]{scatter_f128}\\
 & {\sf 117 Ellipsoid} & {\sf 120 Sum of diff.\ powers} & {\sf 123 Schaffer F7} & {\sf 126 Griewank-Rosenbrock} & {\sf 129 Gallagher}\\
\begin{turn}{90}\parbox{0.175\textwidth}{\centering\sf uniform noise}\end{turn} &
    \includegraphics[height=0.175\textwidth, trim= 34mm 9mm 20mm 9mm, clip]{scatter_f117}&
    \includegraphics[height=0.175\textwidth, trim= 34mm 9mm 20mm 9mm, clip]{scatter_f120}&
    \includegraphics[height=0.175\textwidth, trim= 34mm 9mm 20mm 9mm, clip]{scatter_f123}&
    \includegraphics[height=0.175\textwidth, trim= 34mm 9mm 20mm 9mm, clip]{scatter_f126}&
    \includegraphics[height=0.175\textwidth, trim= 34mm 9mm 20mm 9mm, clip]{scatter_f129}\\
& {\sf 118 Ellipsoid} & {\sf 121 Sum of diff.\ powers} & {\sf 124 Schaffer F7} & {\sf 127 Griewank-Rosenbrock} & {\sf 130 Gallagher}\\
\begin{turn}{90}\parbox{0.175\textwidth}{~\centering\sf Cauchy noise}\end{turn} &
    \includegraphics[height=0.175\textwidth, trim= 34mm 9mm 20mm 9mm, clip]{scatter_f118}&
    \includegraphics[height=0.175\textwidth, trim= 34mm 9mm 20mm 9mm, clip]{scatter_f121}&
    \includegraphics[height=0.175\textwidth, trim= 34mm 9mm 20mm 9mm, clip]{scatter_f124}&
    \includegraphics[height=0.175\textwidth, trim= 34mm 9mm 20mm 9mm, clip]{scatter_f127}&
    \includegraphics[height=0.175\textwidth, trim= 34mm 9mm 20mm 9mm, clip]{scatter_f130}
\end{tabular}
\caption{\label{fig:scatterplots}Expected running time (ERT in
 log10 of number of function evaluations) of \algone\ versus \algzero\ for 46
 target values $\Df \in [10^{-8}, 10]$ in each dimension for functions
 $f_{101}$--$f_{130}$. Markers on the upper or right egde indicate that the target
 value was never reached by \algone\ or \algzero\ respectively. Markers 
 represent dimension: 
  2:{\color{cyan}+},
  3:{\color{green!45!black}$\triangledown$},
  5:{\color{blue}$\star$}, 
 10:$\circ$, 
 20:{\color{red}$\Box$}, 
 40:{\color{magenta}$\Diamond$}. }
\end{figure*}
%%%%%%%%%%%%%%%%%%%%%%%%%%%%%%%%%%%%%%%%%%%%%%%%%%%%%%%%%%%%%%%%%%%%%%%%%%%%%%%
%%%%%%%%%%%%%%%%%%%%%%%%%%%%%%%%%%%%%%%%%%%%%%%%%%%%%%%%%%%%%%%%%%%%%%%%%%%%%%%
\newcommand{\rot}[2][2.5]{
  \hspace*{-3.5\baselineskip}%
  \begin{rotate}{90}\hspace{#1em}#2
  \end{rotate}}
%%%%%%%%%%%%%%%%%%%%%%%%%%%%%%%%%%%%%%%%%%%%%%%%%%%%%%%%%%%%%%%%%%%%%%%%%%%%%%%
%%%%%%%%%%%%%%%%%%%%%%%%%%%%%%%%%%%%%%%%%%%%%%%%%%%%%%%%%%%%%%%%%%%%%%%%%%%%%%%
\begin{figure*}
 \begin{tabular}{l@{\hspace*{-0.025\textwidth}}l|l@{\hspace*{-0.025\textwidth}}l}
 \multicolumn{2}{c}{5-D} & \multicolumn{2}{c}{20-D} \\
 \rot{all functions}
 \includegraphics[width=0.268\textwidth]{pprldistr2_dim05nzall} & 
 \includegraphics[width=0.2375\textwidth,trim=2.3cm 0mm 0mm 0mm, clip]{pplogabs_dim05nzall} &
 \includegraphics[width=0.268\textwidth]{pprldistr2_dim20nzall} &
 \includegraphics[width=0.2375\textwidth,trim=2.3cm 0mm 0mm 0mm, clip]{pplogabs_dim20nzall} \\
 \rot{moderate noise}
 \includegraphics[width=0.268\textwidth,trim=     0mm 0mm 0mm 15mm, clip]{pprldistr2_dim05nzmod} & 
 \includegraphics[width=0.2375\textwidth,trim=2.3cm 0mm 0mm 15mm, clip]{pplogabs_dim05nzmod} &
 \includegraphics[width=0.268\textwidth,trim=     0mm 0mm 0mm 15mm, clip]{pprldistr2_dim20nzmod} &
 \includegraphics[width=0.2375\textwidth,trim=2.3cm 0mm 0mm 15mm, clip]{pplogabs_dim20nzmod} \\
 \rot{severe noise}
 \includegraphics[width=0.268\textwidth,trim=     0mm 0mm 0mm 15mm, clip]{pprldistr2_dim05nzsev} & 
 \includegraphics[width=0.2375\textwidth,trim=2.3cm 0mm 0mm 15mm, clip]{pplogabs_dim05nzsev} &
 \includegraphics[width=0.268\textwidth,trim=     0mm 0mm 0mm 15mm, clip]{pprldistr2_dim20nzsev} & 
 \includegraphics[width=0.2375\textwidth,trim=2.3cm 0mm 0mm 15mm, clip]{pplogabs_dim20nzsev}\\
 \rot[0.5]{severe noise multimod.}
 \includegraphics[width=0.268\textwidth,trim=     0mm 0mm 0mm 15mm, clip]{pprldistr2_dim05nzsmm} & 
 \includegraphics[width=0.2375\textwidth,trim=2.3cm 0mm 0mm 15mm, clip]{pplogabs_dim05nzsmm} &
 \includegraphics[width=0.268\textwidth,trim=     0mm 0mm 0mm 15mm, clip]{pprldistr2_dim20nzsmm} &
 \includegraphics[width=0.2375\textwidth,trim=2.3cm 0mm 0mm 15mm, clip]{pplogabs_dim20nzsmm}
 \end{tabular}
% \vspace*{-0.2cm}
 \caption{\label{fig:RLDs}Empirical cumulative distributions (ECDF)
 of run lengths and speed-up ratios in 5-D (left) and 20-D (right). 
 Left sub-columns: ECDF of
 the number of necessary function evaluations divided by dimension $D$
 (FEvals/D) to reached a target value $\fopt+\Df$ with $\Df=10^{k}$, where
 $k\in\{1, -1, -4, -8\}$ is given by the first value in the legend, for
 \algone\ (solid) and \algzero\ (dashed). Light beige lines show the ECDF of
 FEvals for target value $\Df=10^{-8}$ of all algorithms benchmarked during
 BBOB-2009. 
 Right sub-columns: ECDF of FEval ratios of \algone\ divided by \algzero, all
 trial pairs for each function. Pairs where both trials failed are disregarded,
 pairs where one trial failed are visible in the limits being $>0$ or $<1$. The
 legends indicate the number of functions that were solved in at least one trial
 (\algone\ first).}
\end{figure*}
% \end{document}
%%%%%%%%%%%%%%%%%%%%%%%%%%%%%%%%%%%%%%%%%%%%%%%%%%%%%%%%%%%%%%%%%%%%%%%%%%%%%%%
%%%%%%%%%%%%%%%%%%%%%%%%%%%%%%%%%%%%%%%%%%%%%%%%%%%%%%%%%%%%%%%%%%%%%%%%%%%%%%%
\begin{table*}
\centering
\hfill5-D\hfill20-D\hfill~\\[1ex]
\tiny
\input{\bbobdatapath cmptable_05D_nzall}
\input{\bbobdatapath cmptable_20D_nzall}
\vspace*{-2.5ex}
\caption{\label{tab:ERTs} 
ERT in number of function evaluations 
divided by the best ERT measured during BBOB-2009 (given in the respective
first row) for different $\Df$ values for functions 
$f_{101}$--$f_{130}$. 
\#succ is the number of trials that reached the final target $\fopt + 10^{-8}$.
0:\:\algzeroshort\ is \algzero\ and 1:\:\algoneshort\ is \algone. 
%
\#succ is the number of trials that reached the final target $\fopt + 10^{-8}$.
0:\:\algzeroshort\ is \algzero\ and 1:\:\algoneshort\ is \algone. 
%
Bold entries are statistically significantly better compared 
to the other algorithm, with $p=0.05$ or $p=10^{-k}$ where $k>1$ is the number 
following the $\star$ symbol, with Bonferroni
correction of 60. 
% The first marker compares to the best in 2009, the second marker to 0:\:\algzeroshort.
}
% command definition of \algxxxshort ends here
\end{table*}
%%%%%%%%%%%%%%%%%%%%%%%%%%%%%%%%%%%%%%%%%%%%%%%%%%%%%%%%%%%%%%%%%%%%%%%%%%%%%%%
%%%%%%%%%%%%%%%%%%%%%%%%%%%%%%%%%%%%%%%%%%%%%%%%%%%%%%%%%%%%%%%%%%%%%%%%%%%%%%%

%
% The following two commands are all you need in the
% initial runs of your .tex file to
% produce the bibliography for the citations in your paper.
\bibliographystyle{abbrv}
\bibliography{bbob}  % bbob.bib is the name of the Bibliography in this case
% You must have a proper ".bib" file
%  and remember to run:
% latex bibtex latex latex
% to resolve all references
% to create the ~.bbl file.  Insert that ~.bbl file into
% the .tex source file and comment out
% the command \texttt{{\char'134}thebibliography}.
%
% ACM needs 'a single self-contained file'!
%

% \clearpage % otherwise the last figure might be missing
\end{document}

%%%%%%%%%%%%%%%%%%%%%%%%%%%%%%%%%%%%%%%%%%%%%%%%%%%%%%%%%%%%%%%%%%%%%%%%%%%%%%%%%%%%%%%%%%%
